\documentclass[12pt]{article}
\usepackage{amscd, amssymb, mathtools}
\usepackage{enumerate}

% Palatino for main text and math
\usepackage[osf,sc]{mathpazo}

% Helvetica for sans serif
% (scaled to match size of Palatino)
\usepackage[scaled=0.90]{helvet}

% Bera Mono for monospaced
% (scaled to match size of Palatino)
\usepackage[scaled=0.85]{beramono}

% To start with an epigraph
\usepackage{epigraph}

% %\epigraphsize{\small}% Default
\setlength\epigraphwidth{8cm}
\setlength\epigraphrule{0pt}

\usepackage{etoolbox}

\makeatletter
\patchcmd{\epigraph}{\@epitext{#1}}{\itshape\@epitext{#1}}{}{}
\makeatother
%%

\usepackage{hyperref}



\begin{document}
%
\parskip = 2mm
\begin{center}
{\bf\Large CMP 2020: Preliminary Challenges}

\vspace{3mm}

{\large\bf Anthony Bordg}
\vspace {3mm}

{21 st January 2020}  
\end{center}

\begin{enumerate}
	\item Define Lie groups
	\item Define Lie subgroups
	\item Define the type of representations of a group
	\item State theorem 2.8 in Kirillov
	\item State corollary 2.9 in Kirillov
	\item Define the Lie algebra of a Lie group
	\item Define the exponential map of a Lie group
	\item State Schur's lemma
	\item Define the differential forms on a manifold.
	\item Define the exterior derivative
	\item Define the Hodge star product.
\end{enumerate}

There are at most two places for the summer paid internship. However, every significant contributor of the project will have co-authorship of the resulting publication(s) even if she is not selected for the summer internship (she can still take part remotely anyway).


\end{document}